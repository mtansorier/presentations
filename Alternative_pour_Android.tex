% Copyright (C)  2018  TANSORIER.
% Permission is granted to copy, distribute and/or modify this document
% under the terms of the GNU Free Documentation License, Version 1.3
% or any later version published by the Free Software Foundation;
% with no Invariant Sections, no Front-Cover Texts, and no Back-Cover Texts.
% A copy of the license is included in the section entitled "GNU
% Free Documentation License".

% https://www.gnu.org/licenses/fdl-1.3.html

% compress option to have horyzontal circle
\documentclass[aspectratio=169]{beamer}

%%%%%%%%%%%%%%%%%%%%%%%%%%%%%%%%%%%%%%%%%%%%%%%%%%%%%%%%%%%%%%%%%%%%%

% Thèmes
\usetheme{Darmstadt}

% Redéfini la couleur de base du document
\definecolor{cvp}{RGB}{179,0,161}
\setbeamercolor*{structure}{fg=cvp}

% Language
\usepackage[french]{babel}

% Pour les hyperliens
\usepackage{hyperref}

% Pour rayer un texte avec \sout{}
\usepackage{ulem}

% Pour les icones awesome
\usepackage{fontawesome}

%%%%%%%%%%%%%%%%%%%%%%%%%%%%%%%%%%%%%%%%%%%%%%%%%%%%%%%%%%%%%%%%%%%%%
\title[U-Boot]{Café Vie Privée \\ \textbf{Alternative pour Android}}
\titlegraphic{\includegraphics[width=.5\textwidth]{logos/CVP-Nantes.png}}

\author[Café Vie Privée]{Café Vie Privée}

\date[Août 2018]{\textit{Présentation d'outils libres pour se protéger sur Android}}
%%%%%%%%%%%%%%%%%%%%%%%%%%%%%%%%%%%%%%%%%%%%%%%%%%%%%%%%%%%%%%%%%%%%%

% Pour enregistrer l'écran
% ========================
% (activer le debugage par usb pour accèder vie adb)
% adb shell screenrecord --output-format=h264 - | ffplay -

% Pour streamer l'écran
% =====================
% Utiliser l'application ScreenSream disponible sur F-droid

\begin{document}

% *******************************
% ****     PAGE DE GARDE     ****
% *******************************

\begin{frame}
\titlepage
\end{frame}

% Ajout le logo sur les diapos après la page de garde
\logo{\includegraphics[width=.20\textwidth]{logos/CVP-Nantes.png}}

% *******************************
% ****      INTRODUCTION     ****
% *******************************

\begin{frame}{Déroulé}
\tableofcontents[hideallsubsections]
\end{frame}

\section{Pourquoi se protéger ?}

\begin{frame}
\begin{center}
\huge{\color{cvp}{Pourquoi se protéger ?}}
\end{center}
\end{frame}

\subsection{Qu'est-ce que Android}

\begin{frame}{}
\begin{block}{Wikipedia}
	\emph{\href{https://fr.wikipedia.org/wiki/Android}{Android}} est un système d'exploitation mobile fondé sur le noyau Linux et développé actuellement par Google.
\end{block}
Android à été lancé en 2007.\newline
Le code est sous \emph{licence apache (ASL)}. Il peut donc être intégré dans des applications propriétaires.
\end{frame}

\begin{frame}{Quelques infos}
On peut le trouver sur
\begin{itemize}
	\item Téléphone
	\item Tablette
	\item Télévision
	\item Livre électronique (ex: chomebook)
	\item Montre
	\item Voiture
	\item …
\end{itemize}
Il possède plus de 82\% de part de marché des smartphone (en 2019).
\end{frame}

\subsection{La surveillance de masse}

\begin{frame}
\begin{center}
\large{\color{cvp}{La surveillance de masse}}
\end{center}
\end{frame}

\begin{frame}{Les puissants - la centralisation}
Nos environnements informatiques utilisent des \textbf{logiciels propriétaires},\newline
et pour la plupart détenue par quelques uns qui ont un \textbf{poids important} dans notre société. \newline
\newline
D'où la création du terme \textbf{GAFAM}: Google Apple Facebook Amazon Microsoft\newline
\newline
\textcolor{gray}{\tiny{\sout{GAFAM} $\to$ les vampires de données}}
\end{frame}

\subsection{Économie de la surveillance}

\begin{frame}
\begin{center}
\large{\color{cvp}{Économie de la surveillance}}
\end{center}
\end{frame}

\begin{frame}{Les Publicitaires}
Les applications dépendent d'un model économique, qui souvent est dépendant des publicitaires.\newline
\newline
Pour cela ils utilisent des \texttt{pisteurs}.
\end{frame}

\begin{frame}{Exodus Privacy}
\begin{block}{Exodus Privacy}
Est une plate-forme d'analyse des applications Android qui liste les traqueurs et les autorisations de celles-ci.
\end{block}
Créer en 2017.\newline
\newline
\newline
\textcolor{gray}{\tiny{(ClassyShark3xodus le fait pour d'autres apk du système)}}
\end{frame}

\subsection{Programme de surveillance}

\begin{frame}
\begin{center}
\large{\color{cvp}{Programme de surveillance}}
\end{center}
\end{frame}

\begin{frame}{Les enjeux}
Il existe de nombreux programmes de surveillance plus ou moins connus.
\begin{block}{Jérémy Zimmermamn}
Ne pas confondre entre: j'ai rien à me reprocher et j'ai rien à cacher.
\end{block}
Cf. Le documentaire Notiong To Hide \textcolor{gray}{\tiny{(\url{https://vimeo.com/193515863})}}
\end{frame}

\begin{frame}{}
Refuser les programmes de surveillance des données comme PRISM, XKeyscore et Tempora.
% PRISM: Progamme de surveillanc américan. Collecte les données sur internet.
%	https://fr.wikipedia.org/wiki/PRISM_(programme_de_surveillance)
% XKeyscore: Programme de surveillance créer par la NSA. Collecte quasi systématique des activités de tout utilisateur sur Internet.
%            Révélé en juillet 2013.
% 	https://fr.wikipedia.org/wiki/XKeyscore
% Tempora: Programme britanique permettant d'intercepter les données transitant par les câbles en fibre optique entre l'europe et les États-Unis.

\begin{tiny}
\begin{block}{PRISM}
Progamme de surveillance américan créer par la NSA. Collecte les données sur internet.
Edward Snowden a dénoncé ce programme en juin 2013.
\end{block}

\begin{block}{XKeyscore}
Programme de surveillance créé par la NSA. Collecte quasi systématique des activités de tout utilisateur sur Internet. Révélé en juillet 2013.
\end{block}

\begin{block}{Tempora}
Programme britanique permettant d'intercepter les données transitant par les câbles en fibre optique entre l'Europe et les États-Unis.
\end{block}
\end{tiny}
\end{frame}

\begin{frame}{Contre attaque}
Il existe des projet pour se défendre contre ça:
\url{https://prism-break.org/fr/categories/android/}\newline
\newline
On y reviendra :)
\end{frame}


% *******************************
% ****  COMMENT SE PROTEGER  ****
% *******************************

\section{Comment se protéger ?}

\begin{frame}
\begin{center}
\huge{\color{cvp}{Comment se protéger ?}}
\end{center}
\end{frame}

\subsection{Avec du logiciel Libre}

\begin{frame}
Accéder au code source permet d'\textbf{auditer} le code et vérifier que le logiciel correspond à ce pourquoi il est fait.\footnote{Confiance par la transparence $ \neq $ Sécurité pas l'obscurité.} \newline
\newline
Le logiciel libre permet de \textbf{modifier} le code pour nos besoins personnels.\newline
\newline
Ainsi que d'installer librement le logiciel.\newline
\textcolor{gray}{\tiny{On peut appeler ça la \textbf{décentralisation} pour des serveurs, c'est ce qui permet l'autonomie.}}\newline
\newline
De faite, le logiciel libre ne \textbf{dépend pas} d'un \textbf{modèle économique} particulier.
\end{frame}

\subsection{Avec quels logiciels ?}

\begin{frame}
On va voir quelques exemples:\newline
\begin{itemize}
	\item Magasin d'applications
	\item Navigation
	\item Réseaux sociaux
	\item Cartes
	\item Multimédia
	\item Recherches
	\item Divers
\end{itemize}
\end{frame}


% *******************************
% ****   LES APPLICATIONS    ****
% *******************************

\section{Les Applications}

\begin{frame}
\begin{center}
\huge{\color{cvp}{Les Applications}}
\end{center}
\end{frame}

\subsection{Magasin d'application}

\begin{frame}
Playstore $\to$ F-Droid\newline
Playstore $\to$ Aurora Droid\newline
\newline
Playstore $\to$ Yalp store\newline
Playstore $\to$ Aurora Store
\end{frame}

\subsection{Navigation}
\begin{frame}
Chrome $\to$ Firefox\newline
Google $\to$ Duckduckgo\newline
\newline
Tor Browser \textcolor{gray}{\tiny{(anciennement Orfox)}}\newline
\newline
Orbot \textcolor{gray}{\tiny{(Proxy pour le réseau Tor)}}
\end{frame}

\subsection{Réseaux sociaux}
\begin{frame}
Facebook $\to$ slim social \textcolor{gray}{\tiny{(ou Frost for facebook)}}\newline
\newline
Diaspora\newline
Mastodon
\end{frame}

\subsection{Messagerie instantanée}
\begin{frame}
Whatsapp $\to$ Signal\newline
SMS $\to$ Silence\newline
Conversation \textcolor{gray}{\tiny{(XMPP)}}
\end{frame}

\subsection{Cartes}
\begin{frame}
OsmAnd \textcolor{gray}{\tiny{(Openstreetmaps)}}
\end{frame}

\subsection{multimédia}
\begin{frame}
Youtube $\to$ NewPipe\newline
Peertube $\to$ Thoruim
\end{frame}

\subsection{Recheche}
\begin{frame}
Google $\to$ Duckduckgo
\end{frame}

\subsection{Divers}
\begin{frame}
\begin{description}
	\item[Clavier] Google $\to$ AnySoftKey
	\item[Météo] Votre météo locale \textcolor{gray}{\tiny{(F-droid)}}
	\item[Caméra] Open Caméra
	\item[Transport] Transportr
	\item[Bloqueur] Blokada
	\item[Liseuse] Book Reader
	\item[Luminosité] Red Moon
	\item[Battery] Drowser
\end{description}
\end{frame}

% *******************************
% ****     OS ALTERNATIF     ****
% *******************************

\section{OS Alternatif}
% https://linuxfr.org/news/installer-lineageos-sur-son-appareil-android

\begin{frame}
\begin{center}
\huge{\color{cvp}{OS Alternatif}}
\end{center}
\end{frame}

\subsection{État de l'art}
\begin{frame}
Il n'existe pas beaucoup d'alternatives complètes, car la difficulté est le support matériel du téléphone. (Il n'existe pas de matériel Libre.)\newline
\newline
Exemple:
\begin{itemize}
	\item \href{https://www.replicant.us}{Replicant}
	\item \href{https://www.pureos.net}{PureOS}
	\item \href{https://e.foundation}{/e/}
	\item \href{https://www.plasma-mobile.org}{PlasmaMobile}
	\item \href{https://lineageos.org}{LineageOS}
\end{itemize}
\end{frame}

\subsection{Replicant}
\begin{frame}
\href{https://fr.wikipedia.org/wiki/Replicant_\%28syst\%C3\%A8me_d\%27exploitation\%29}{\textbf{Replicant}} est un système d'exploitation \textbf{entièrement libre} pour les smartphones et les tablettes, en remplaçant les composants privateurs d'Android par leurs équivalents libres.
\end{frame}

\subsection{PureOS}
\begin{frame}
\href{https://puri.sm/products/librem-5/pureos-mobile/}{\textbf{PureOS}} est un OS (non-Android) pour le téléphone \href{https://puri.sm/products/librem-5/}{Librem5} fabriqué par Purism.\newline
\newline
L'OS est un système GNU/Linux.\newline
Peut se lier avec KDE et Gnome.
\end{frame}

\subsection{/e/}
\begin{frame}
\textbf{/e/} une entreprise faisant un OS pour smatphone (éponyme) basé sur Android, sans la sur-couche google.\newline
\newline
Leur \href{https://e.foundation/wp-content/uploads/ManifesteFR.pdf}{Manifeste} décrit leur démarche.
\end{frame}

\subsection{Plasma-mobile}
\begin{frame}
\textbf{Plasma Mobile} est un OS basé sur GNU/Linux, fait pour fonctionner avec gestionnaire de fenêtre Linux KDE.
\end{frame}

\subsection{LineageOS}

\begin{frame}
\textbf{LineageOS} est un OS pour les téléphones et tablettes basé sur Android Open Source Project (AOSP).\newline
\newline
C'est un fork de Cyanogenmod créer en décembre 2016.
\end{frame}

\begin{frame}{Un système «classique»}
% Documentation:
% https://linuxfr.org/news/installer-lineageos-sur-son-appareil-android
% https://www.montelephonelibre.fr
\begin{center}
\includegraphics[width=.75\textwidth]{schemas/Android_private.png}
\end{center}
\textcolor{gray}{\tiny{Image tiré de \url{https://www.montelephonelibre.fr}}}
\end{frame}

\begin{frame}{Un système plus libre}
\begin{center}
\includegraphics[width=.75\textwidth]{schemas/Android_free.png}
\end{center}
\textcolor{gray}{\tiny{Image tiré de \url{https://www.montelephonelibre.fr}}}
\end{frame}

\subsection{MicroG}
\begin{frame}
LineageOS permet d'utiliser son système avec \textbf{MircoG}.\newline
\end{frame}

\subsection{Mise en pratique}
\begin{frame}
\begin{center}
\includegraphics[width=.50\textwidth]{images/retro-phone.jpg}
\end{center}
\end{frame}

\end{document}
